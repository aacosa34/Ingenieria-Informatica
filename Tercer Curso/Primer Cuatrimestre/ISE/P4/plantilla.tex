\input{preambulo.tex}

%----------------------------------------------------------------------------------------
%	TÍTULO Y DATOS DEL ALUMNO
%----------------------------------------------------------------------------------------

\title{	
\normalfont \normalsize 
\textsc{\textbf{Asignatura (2021-2022)} \\ Grado en Ingeniería Informática \\ Universidad de Granada} \\ [25pt] % Your university, school and/or department name(s)
\horrule{0.5pt} \\[0.4cm] % Thin top horizontal rule
\huge Memoria Práctica 3 \\ % The assignment title
\horrule{2pt} \\[0.5cm] % Thick bottom horizontal rule
}

\author{Pedro Antonio Mayorgas Parejo} % Nombre y apellidos

\date{\normalsize\today} % Incluye la fecha actual

%----------------------------------------------------------------------------------------
% DOCUMENTO
%----------------------------------------------------------------------------------------

\begin{document}

\maketitle % Muestra el Título

\newpage %inserta un salto de página

\tableofcontents % para generar el índice de contenidos
 
\newpage

%----------------------------------------------------------------------------------------
%	Cuestión 1
%----------------------------------------------------------------------------------------

\section{Seccion}


%\begin{lstlisting}[language=bash,alsoletter=/,frame=single]
%	mdadm -D [dispositivo]
%\end{lstlisting}


%\begin{figure}[H]
%	\centering
%	\includegraphics[scale=0.30]{cuestion_1_1}
%	\caption{Se puede ver que al no haber un fallo grave, el sistema lo nota como que sigue funcionando pero en un estado degradado.}
%\end{figure}
 
%\newpage

%Se pueden hacer include en latex
\input{plantilla_include.tex}


%-------Bibliografia-----------------------------

\newpage
\section{Bibliografía}

\footnote{Instalación y configuración de Zabbix server en Rocky Linux 8 o CentOS8}
\textcolor{blue}{\url{https://techviewleo.com/install-and-configure-zabbix-server-on-rocky-linux/}}

\footnote{Instalación de Zabbix server en Rocky Linux 8 o CentOS8}
\textcolor{blue}{\url{https://www.zabbix.com/download?zabbix=5.0\&os_distribution=centos&os_version=8\&db=mysql\&ws=apache}}

\footnote{Gestión de RAID y reparación de fallos}
\textcolor{blue}{\url{https://access.redhat.com/documentation/en-\%20us/red\_hat\_enterprise\_linux/8/html/managing\_storage\_devices/managing-raid\_managing-storage-devices}}

\footnote{Instalación de Ansible en Rocky Linux 8}
\textcolor{blue}{\url{https://www.how2shout.com/linux/how-to-install-ansible-on-rocky-linux-8-or-almalinux/}}

\footnote{Conceptos básicos de Ansible}
\textcolor{blue}{\url{https://www.redhat.com/es/topics/automation/learning-ansible-tutorial}}

\footnote{Instalación de EPEL en CentOS8 o Rocky Linux 8}
\textcolor{blue}{\url{https://fedoraproject.org/wiki/EPEL/es}}

\footnote{Página del manual de linux sobre Ansible}
\textcolor{blue}{\url{https://linux.die.net/man/1/ansible}}



\end{document}
