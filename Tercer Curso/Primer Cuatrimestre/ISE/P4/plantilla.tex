%----------------------------------------------------------------------------------------
%	PACKAGES AND OTHER DOCUMENT CONFIGURATIONS
%----------------------------------------------------------------------------------------

\documentclass[paper=a4, fontsize=11pt]{scrartcl} % A4 paper and 11pt font size

% ---- Entrada y salida de texto -----

\usepackage[T1]{fontenc} % Use 8-bit encoding that has 256 glyphs
\usepackage[utf8]{inputenc}
%\usepackage{fourier} % Use the Adobe Utopia font for the document - comment this line to return to the LaTeX default

% ---- Idioma --------

\usepackage[spanish, es-tabla]{babel} % Selecciona el español para palabras introducidas automáticamente, p.ej. "septiembre" en la fecha y especifica que se use la palabra Tabla en vez de Cuadro

% ---- Otros paquetes ----
\usepackage{csquotes} %Para permitir el uso de comillas Quotes https://tex.stackexchange.com/questions/36812/isnt-there-any-other-way-of-doing-double-quotes-in-latex-besides
\usepackage[hyphens]{url} % ,href} %para incluir URLs e hipervínculos dentro del texto (aunque hay que instalar href)
\usepackage{hyperref}
\usepackage{color}
\usepackage{graphics,graphicx, floatrow} %para incluir imágenes y notas en las imágenes
\usepackage{graphics,graphicx, float} %para incluir imágenes y colocarlas

\graphicspath {{./img/}}

\usepackage{listings}  %para introducir comandos
\lstset{basicstyle=\ttfamily,
  showstringspaces=false,
  commentstyle=\color{red},
  keywordstyle=\color{blue}
}
% Para hacer tablas comlejas
%\usepackage{multirow}
%\usepackage{threeparttable}

%\usepackage{sectsty} % Allows customizing section commands
%\allsectionsfont{\centering \normalfont\scshape} % Make all sections centered, the default font and small caps

\usepackage{fancyhdr} % Custom headers and footers
\pagestyle{fancyplain} % Makes all pages in the document conform to the custom headers and footers
\fancyhead{} % No page header - if you want one, create it in the same way as the footers below
\fancyfoot[L]{} % Empty left footer
\fancyfoot[C]{} % Empty center footer
\fancyfoot[R]{\thepage} % Page numbering for right footer
\renewcommand{\headrulewidth}{0pt} % Remove header underlines
\renewcommand{\footrulewidth}{0pt} % Remove footer underlines
\setlength{\headheight}{13.6pt} % Customize the height of the header

\setlength\parindent{0pt} % Removes all indentation from paragraphs - comment this line for an assignment with lots of text

\newcommand{\horrule}[1]{\rule{\linewidth}{#1}} % Create horizontal rule command with 1 argument of height


%----------------------------------------------------------------------------------------
%	TÍTULO Y DATOS DEL ALUMNO
%----------------------------------------------------------------------------------------

\title{	
\normalfont \normalsize 
\textsc{\textbf{Asignatura (2021-2022)} \\ Grado en Ingeniería Informática \\ Universidad de Granada} \\ [25pt] % Your university, school and/or department name(s)
\horrule{0.5pt} \\[0.4cm] % Thin top horizontal rule
\huge Memoria Práctica 3 \\ % The assignment title
\horrule{2pt} \\[0.5cm] % Thick bottom horizontal rule
}

\author{Pedro Antonio Mayorgas Parejo} % Nombre y apellidos

\date{\normalsize\today} % Incluye la fecha actual

%----------------------------------------------------------------------------------------
% DOCUMENTO
%----------------------------------------------------------------------------------------

\begin{document}

\maketitle % Muestra el Título

\newpage %inserta un salto de página

\tableofcontents % para generar el índice de contenidos
 
\newpage

%----------------------------------------------------------------------------------------
%	Cuestión 1
%----------------------------------------------------------------------------------------

\section{Seccion}


%\begin{lstlisting}[language=bash,alsoletter=/,frame=single]
%	mdadm -D [dispositivo]
%\end{lstlisting}


%\begin{figure}[H]
%	\centering
%	\includegraphics[scale=0.30]{cuestion_1_1}
%	\caption{Se puede ver que al no haber un fallo grave, el sistema lo nota como que sigue funcionando pero en un estado degradado.}
%\end{figure}
 
%\newpage

%Se pueden hacer include en latex
\newpage

\section{Section}

\subsection{Subseccion}

\subsubsection{Subseccion}



%-------Bibliografia-----------------------------

\newpage
\section{Bibliografía}

\footnote{Instalación y configuración de Zabbix server en Rocky Linux 8 o CentOS8}
\textcolor{blue}{\url{https://techviewleo.com/install-and-configure-zabbix-server-on-rocky-linux/}}

\footnote{Instalación de Zabbix server en Rocky Linux 8 o CentOS8}
\textcolor{blue}{\url{https://www.zabbix.com/download?zabbix=5.0\&os_distribution=centos&os_version=8\&db=mysql\&ws=apache}}

\footnote{Gestión de RAID y reparación de fallos}
\textcolor{blue}{\url{https://access.redhat.com/documentation/en-\%20us/red\_hat\_enterprise\_linux/8/html/managing\_storage\_devices/managing-raid\_managing-storage-devices}}

\footnote{Instalación de Ansible en Rocky Linux 8}
\textcolor{blue}{\url{https://www.how2shout.com/linux/how-to-install-ansible-on-rocky-linux-8-or-almalinux/}}

\footnote{Conceptos básicos de Ansible}
\textcolor{blue}{\url{https://www.redhat.com/es/topics/automation/learning-ansible-tutorial}}

\footnote{Instalación de EPEL en CentOS8 o Rocky Linux 8}
\textcolor{blue}{\url{https://fedoraproject.org/wiki/EPEL/es}}

\footnote{Página del manual de linux sobre Ansible}
\textcolor{blue}{\url{https://linux.die.net/man/1/ansible}}



\end{document}
